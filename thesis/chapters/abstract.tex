% !TEX root = ../main_standard.tex
\chapter*{\Large \center Abstract}
Previous work on arthropod animation can be divided into two main categories: data driven approach and physical modeling approach. The locomotion of various arthropods has been extensively studied in biology and robotics. A simulation of arthropod could be used in various meaningful scenarios such as animation movies, computer games, etc.



The main aim of the project is to develop realistic spider animation. A subsidiary aim is to enable people to be immersed in the animation by using the Oculus Rift. The purpose of this report is to review general animation techniques, knowledge for simulating an arthropod and finally come up with possible solutions for spider animation. The general animation adopted for the project is procedural animation. Two main approaches proposed to control the animation are: scene graph and hierarchy with rigging and skinning.


