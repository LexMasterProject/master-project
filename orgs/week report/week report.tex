% Created 2015-07-20 Mon 11:31
\documentclass[11pt]{article}
\usepackage[utf8]{inputenc}
\usepackage[T1]{fontenc}
\usepackage{fixltx2e}
\usepackage{graphicx}
\usepackage{longtable}
\usepackage{float}
\usepackage{wrapfig}
\usepackage{rotating}
\usepackage[normalem]{ulem}
\usepackage{amsmath}
\usepackage{textcomp}
\usepackage{marvosym}
\usepackage{wasysym}
\usepackage{amssymb}
\usepackage{hyperref}
\tolerance=1000
\author{Wuhao Wei}
\date{\today}
\title{July 20 week report}
\hypersetup{
  pdfkeywords={},
  pdfsubject={},
  pdfcreator={Emacs 24.5.1 (Org mode 8.2.10)}}
\begin{document}

\maketitle
\tableofcontents


\section{Work done:}
\label{sec-1}


\subsection{Cameras}
\label{sec-1-1}
press v to switch cameras

\subsubsection{auto follow camera}
\label{sec-1-1-1}

\subsubsection{controllable follow camera}
\label{sec-1-1-2}
operation instructions:
\begin{enumerate}
\item use > or < key to change the dist between the spider and the camera
\item use w/a/s/d to rotate
\end{enumerate}

\subsubsection{world camera}
\label{sec-1-1-3}
There are two kinds of world cameras.
\begin{enumerate}
\item First one is just operating the cameras.
\item Second one is cast a ray from the camera to gameobjects.
\end{enumerate}
The difference is the rotating operations:
\begin{enumerate}
\item the first one rotates the camera itself
\item the second one rotates the camera around the ray hit point.
\end{enumerate}
Cons\&pros:
\begin{enumerate}
\item The first one is hard to operate.
\item The second one is easy to operate. Besides, I could shoot food or bombs towards the hit point. 
However, the ray must hit at least one objects. And one problem here is how to deal with the condition that the ray hit nothing.
\end{enumerate}
operation instructions:
\begin{enumerate}
\item 1,2,3,4 to reset the camera position and orientation
\item w/s/a/d to rotate
\item > or < to change the dist between hit point and camera
\item arrow keys to change the position of the camera
\end{enumerate}

Additionally, in world camera state, press f to place a fly.

\subsection{add different behaviours}
\label{sec-1-2}
run towards after the spider sense a fly
\subsection{add simple UI}
\label{sec-1-3}


\section{Report revision}
\label{sec-2}
have done some some revision. 
add footnote. need some suggestion on the permission part.
e.g. 
\begin{figure}[htb]
\centering
\includegraphics[width=.9\linewidth]{./img/permission_sample.png}
\caption{\label{mylabel}permission.}
\end{figure}

\begin{figure}[htb]
\centering
\includegraphics[width=.9\linewidth]{./img/how_to_add.png}
\caption{\label{mylabel}bg$_{\text{report}}$.}
\end{figure}




\section{Extra reading}
\label{sec-3}
\begin{enumerate}
\item read materials about Behaviour-based robotics
subsumption architecture(most reading are slides and blogs)
\item Brooks, R. (1986). A robust layered control system for a mobile robot
\item Rodney Brooks’s paper on “Fast, Cheap, and Out of Control”
\end{enumerate}


\section{Problems}
\label{sec-4}
\begin{enumerate}
\item understand behaviour-based and subsumption architecture. 
Still need some extra efforts on implementation
(Since the robotics works differently. e.g. aims, difference between animation and
robotics, hardware.).
One big chanllenge here is how to implement parallel layers.(one possible solution 
is to implement a schedule system with para priority)

\item how to make the clips and movement consistent.
\item similar to problem2. Direct control and indirect control.
One way to control the spider is to manipulate:
\begin{center}
\begin{tabular}{lll}
target position & velocity & limb frequency\\
target direction & angular velocity & limb turn\\
\end{tabular}
\end{center}

The other way is 
\begin{center}
\begin{tabular}{lll}
target & transform & limb\\
\end{tabular}
\end{center}

\item action sequence
\end{enumerate}


\section{Todo}
\label{sec-5}

\begin{enumerate}
\item revise the bg report(will mainly focus on chapter2 literature review)
\item add full details(e.g. turn around process. For the moment, cases are simplified. e.g. directly rotate the spider without process)
\item add new behaviours
\item refactor the main code(spider.cs see details in scratch)
\end{enumerate}
% Emacs 24.5.1 (Org mode 8.2.10)
\end{document}